\documentclass[a4paper,10pt]{article}
\usepackage[utf8]{inputenc}
\usepackage{graphicx}
\usepackage{color}
\usepackage{natbib}
\usepackage{pdfpages}
\usepackage{float}
\usepackage{amsmath }

%Journals
\def\apj {ApJ}
\def\apjl {ApJL}
\def\apjs {ApJS}
\def\aj {AJ}
\def\aap {A\&A}
\def\mnras {MNRAS}

% units
\newcommand{\pc}{{\rm pc}}
\newcommand{\kpc}{{\rm kpc}}
\newcommand{\Mpc}{{\rm Mpc}}
\newcommand{\kms}{{\rm km\,s^{-1}}}
\newcommand{\keV}{{\rm keV}}
\newcommand{\s}{{\rm s}}
\newcommand{\Msun}{{\rm M_{\odot}}}
\newcommand{\fdev}{\rm FracDeV_{r}}
%.................................................................................................
%opening
\title{Segregaci\'on Morfol\'ogica en C\'umulos Ricos}
\author{Agustín Rodríguez M.}

\begin{document}

\maketitle

\begin{abstract}

En este trabajo estudiamos la segregaci\'on morfol\'ogica de los c\'umulos de galaxias. Utilizamos la muestra de galaxias en c\'umulos generada por \citet{Dressler1980.catalogo} y reprodujimos los resultados obtenidos por \citet{Dressler1980}. Encontramos que las galaxias espirales, lenticulares y el\'ipticas son sensibles a estos entornos. La fracci\'on de galaxias el\'ipticas y lenticulares crece a medida que aumenta la densidad, y la fracci\'on de espirales decrecen.
Estudiamos la fracci\'on de los tipos morfol\'ogicos en funci\'on de la distancia al centro del c\'umulo, de acuerdo a lo realizado por \citet{Whitmore1993}, donde obtuvimos que la fracci\'on de galaxias el\'ipticas y lenticulares crece hac\'ia el centro de los c\'umulos, y las espirales decrecen. La relaci\'on morfología-centro es mas fundamental, sugiriendo que son las propiedades globales quienes controlan las fracciones morfol\'ogicas en los c\'umulos de galaxias.
\end{abstract}

\citep{Correa2019}
\citep{Springel2001}
\citep{Ruiz2015}
\citep{EAGLE}
\citep{Paillas2017}
\citep{Pisani2015}
\citep{Cai2014}
\citep{Massara2015}
\citep{Hopkins2015}

%.................................................................................................






















\bibliographystyle{aa} %.bst
\bibliography{ref} %.bib

%\begin{thebibliography}{}
%\bibitem[York et al.(2000)]{b1} York, D.~G., Adelman, J., Anderson, J.~E., Jr., et al.\ 2000, \aj, 120, 1579 
%\bibitem[Yip et al.(2004)]{Yip:2004} Yip, C.~W., Connolly, A.~J., Szalay, A.~S., et al.\ 2004, \aj, 128, 585 
%\bibitem[Mart{\'{\i}}nez \& Muriel(2011)]{b2} Mart{\'{\i}}nez, H.~J., \& Muriel, H.\ 2011, \mnras, 418, L148
%\end{thebibliography}

%..................................................................................

\end{document}

